\documentclass{article}

\usepackage{amsmath}
\usepackage[utf8]{inputenc}
\usepackage{babel}
\usepackage{hyperref}
\usepackage{graphicx}
\usepackage{listings}

\usepackage[margin=0.5in]{geometry}

\title{Linear Codes}
\date{\today{}}
\author{typeset by Dave Friedman}

\begin{document}

\maketitle{}
\pagenumbering{gobble}
\newpage{}
\pagenumbering{arabic}
\tableofcontents{}
\setcounter{secnumdepth}{0}
\newpage{}

%%%%%
%
% 
%
%%%%%

\subsubsection{Definition: Inner Product}
The \emph{inner product} $\textbf{u}\cdot\textbf{v}$ of vectors $\textbf{u}=u_1...u_n$ and $\textbf{v}=v_1...v_n$ in $V(n,q)$ is the scalar defined by \[\textbf{u}\cdot\textbf{v}=u_1v_1+...+u_nv_n\]
If $\textbf{u}\cdot\textbf{v}=0$, then $\boldsymbol{u}$ and $\boldsymbol{v}$ are called \emph{orthogonal}.
\subsubsection{Lemma 7.1}

\end{document}
